\documentclass[onecolumn, draftclsnofoot,10pt, compsoc]{IEEEtran}
\usepackage{graphicx}
\usepackage{url}
\usepackage{setspace}

\usepackage{geometry}
\usepackage{mathtools}

\geometry{textheight=9.5in, textwidth=7in}
\geometry{margin=0.75in}

% 1. Fill in these details
\def \CapstoneTeamName{		Team TriTone}
\def \CapstoneTeamNumber{		45}
\def \GroupMemberOne{			Aidan O'Malley}
\def \GroupMemberTwo{			Christopher Hebert}
\def \GroupMemberThree{			Kazuriah Buckley}
\def \CapstoneProjectName{		Music Theory Application}
\def \CapstoneSponsorPerson{		Lukas Hein}

% 2. Uncomment the appropriate line below so that the document type works
\def \DocType{		%Problem Statement
                %Requirements Document
                Technology Review
                %Design Document
                %Progress Report
                }
            
\newcommand{\NameSigPair}[1]{\par
\makebox[2.75in][r]{#1} \hfil 	\makebox[3.25in]{\makebox[2.25in]{\hrulefill} \hfill		\makebox[.75in]{\hrulefill}}
\par\vspace{-12pt} \textit{\tiny\noindent
\makebox[2.75in]{} \hfil		\makebox[3.25in]{\makebox[2.25in][r]{Signature} \hfill	\makebox[.75in][r]{Date}}}}
% 3. If the document is not to be signed, uncomment the RENEWcommand below
%\renewcommand{\NameSigPair}[1]{#1}

%%%%%%%%%%%%%%%%%%%%%%%%%%%%%%%%%%%%%%%
\begin{document}
\begin{titlepage}
    \pagenumbering{gobble}
    \begin{singlespace}
        % \includegraphics[height=4cm]{coe_v_spot1}
        \hfill 
        % 4. If you have a logo, use this includegraphics command to put it on the coversheet.
        %\includegraphics[height=4cm]{CompanyLogo}   
        \par\vspace{.2in}
        \centering
        \scshape{
            \huge CS Capstone \DocType \par
            {\large\today}\par
            \vspace{.5in}
            \textbf{\Huge\CapstoneProjectName}\par
            \vfill
            % {\large Prepared for}\par
            % \Huge \CapstoneSponsorCompany\par
            % \vspace{5pt}
            {\Large\NameSigPair{\CapstoneSponsorPerson}\par}
            {\large Prepared by }\par
            Group\CapstoneTeamNumber\par
            % 5. comment out the line below this one if you do not wish to name your team
            \CapstoneTeamName\par 
            \vspace{5pt}
            {\Large
                \NameSigPair{\GroupMemberOne}\par
                \NameSigPair{\GroupMemberTwo}\par
                \NameSigPair{\GroupMemberThree}\par
            }
            \vspace{20pt}
        }
        \begin{abstract}
        % 6. Fill in your abstract  
            This document is a technology review for the Music Theory Application that Team TriTone will be creating for Lukas Hein.
            This document contains Aidan's review of his three pieces.
            These pieces are the overall development environment, the base circle of fifths implementation, and the circle of fifths sidebar implementation.
        \end{abstract}     
    \end{singlespace}
\end{titlepage}
\newpage
\pagenumbering{arabic}
% \tableofcontents
% 7. uncomment this (if applicable). Consider adding a page break.
%\listoffigures
%\listoftables
% \clearpage

% 8. now you write!
\section{Overall Development Environment}
\subsection{Overview}
Since our team's task is to create an application, the first hurdle that we must tackle is what environment are we going to define as our base to build the application upon.
There are more possibilities than what we are going to discuss in this paper, however, for our problem we have narrowed the best choices for the development environment to Xamarin Studio, XCode and Android Studio to create separate but equivalent products, or React Native.

\subsection{Criteria}
To judge what will determine an acceptable development environment, we need to be able to meet all of the requirements we defined in the requirements document.
The main hurdles in that document will be creating a chart to represent the circle of fifths, adding interaction capabilities to this chart, as well as creating a page to analyze a user's compositions.
The other aspects of the application are mostly simple text based components and should be easily accomplished in any of the development environments we defined.

\subsection{Pieces}
\subsubsection{Xamarin Studio}

Pros

\begin{itemize}
    \item One codebase for both android and ios platforms
    \item Component store
    \item Object oriented code base
    \item Good support and resources
\end{itemize}

Cons

\begin{itemize}
    \item C\# which only one team member has worked with
    \item Difficult user interface creation
\end{itemize}
\subsubsection{XCode and Android Studio}

Pros

\begin{itemize}
    \item Guaranteed full use of mobile's capabilities since we are coding in the intended manner for both platforms
\end{itemize}

Cons

\begin{itemize}
    \item Two codebases would be necessary for the two different languages
    \item No members of the team have worked in swift before
\end{itemize}

\subsubsection{React Native}

Pros

\begin{itemize}
    \item Quick set-up
    \item Javascript is the programming language used which all team members are comfortable with
    \item Npm support
    \item Easy to convert code to a web platform
    \item Good support and resources/documentation
    \item Component based
    \item One codebase for both android and ios platforms
\end{itemize}

Cons

\begin{itemize}
    \item Relatively new and doesn't have all of the built in features that a true development suite like xamarin has
    \item Might not have all of the capabilities that coding in the native environment would allow
\end{itemize}

\subsection{Discussion}
\subsection{Conclusion}
We decided on using React Native as our development environment for many reasons.
One reason being the wide range of free third party libraries.
Another being the fact that it is one codebase for both platforms.
Lastly, there is the least boundary to start with React Native since every team member is comfortable with coding in JavaScript.


\section{Base Circle of Fifths Implementation}
\subsection{Overview}
The main part of our application is going to be the Circle of Fifths page.
This is what demonstrates most of our client's topics that he wants to teach.
It is a circle with the 12 tones of western music around it.
Going clockwise around the circle are the intervals of four and going counter clockwise are the fifth intervals.
There are many ways we could implement this circle but we have narrowed our options down to creating the circle with something similar to a browser SVG (vector based graphic), loading specific images into the app and cycling through them when a user interacts with the app, or using ascii art to make up the circle.

\subsection{Criteria}
The criteria we defined to determine if the circle of fifths is acceptable are listed.
It needs to be easy to make out all of the notes on the circle, it should also be easy to understand the flow from relative key to main key to parallel key on the circle.
The circle should be visually pleasing to the user meaning it matches the rest of the theme and doesn't stand out in a bad way.

\subsection{Pieces}
\subsubsection{SVG}

Pros

\begin{itemize}
    \item More customizability
    \item No reizing issues
\end{itemize}

Cons

\begin{itemize}
    \item Harder set up
    \item More complex code in general because you need to draw out the paths of the image
\end{itemize}

\subsubsection{Image}

Pros

\begin{itemize}
    \item Simple to code interactions with the image
    \item With good vector image software we could make really nice images
\end{itemize}

Cons

\begin{itemize}
    \item Will need $12^4$ images for all of the different options we need
\end{itemize}

\subsubsection{Ascii}

Pros

\begin{itemize}
    \item Easy to code
    \item Easy to manipulate
\end{itemize}

Cons

\begin{itemize}
    \item Looks ugly
\end{itemize}

\subsection{Discussion}
\subsection{Conclusion}
We ended up choosing the SVG option for making the circle of fifths.
One reason for this choice is the customizability.
Our client wants a color coded circle and if we decide later on that we want slick animations involved then all of this is easier witha  drawn SVG in our codebase.
We don't have to bog our repo down with tons of large image files.
We can create the best looking circle with this option.
We deemed that the difficulty of getting the SVG to work in our application to be worth it because of all of the other features we can implement.
Since the circle of fifths is the biggest part of our application we can handle the difficulty of an SVG.

\section{Sidebar Circle of Fifths Implementation}
\subsection{Overview}
Since the main part of our application is the circle of fifths page, it is necessary to talk about our implementation of the sidebar for the circle of fifths.
The sidebar's main purpose is to show the user the schedule of tonal gravity for the key that is currently selected.
The second purpose of the sidebar is to alter the contents of the circle, for example, the sidebar should allow the user to change the current key in some manner.

\subsection{Criteria}
The criteria we defined for a working sidebar is the following.
The user should be able to set the main key with the sidebar.
The user should be able to view the tonal gravity of the current key.

\subsection{Pieces}
\subsubsection{Clickable}

Pros

\begin{itemize}
    \item 
\end{itemize}

Cons

\begin{itemize}
    \item 
\end{itemize}

\subsubsection{Draggable}

Pros

\begin{itemize}
    \item 
\end{itemize}

Cons

\begin{itemize}
    \item 
\end{itemize}

\subsubsection{Combination of Click and Drag}

Pros

\begin{itemize}
    \item Maximizes user's gesture options
\end{itemize}

Cons

\begin{itemize}
    \item Gestures could possibly cancel each other out or interact poorly
\end{itemize}

\subsection{Discussion}
\subsection{Conclusion}
We decided to go with the combination of clicking and dragging to give the user the most interaction with the circle of fifths. 

\end{document}