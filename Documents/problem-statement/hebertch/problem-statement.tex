\documentclass[onecolumn, draftclsnofoot,10pt, compsoc]{IEEEtran}
\usepackage{graphicx}
\usepackage{url}
\usepackage{setspace}

\usepackage{geometry}
\geometry{textheight=9.5in, textwidth=7in}

% 1. Fill in these details
%\def \CapstoneTeamName{		 Cleverly Named Team}
\def \CapstoneTeamNumber{		045}
\def \GroupMemberOne{			Christopher Hebert}
\def \GroupMemberTwo{			Kaz Buckley}
\def \GroupMemberThree{			Aidan O'Malley}
\def \CapstoneProjectName{		Interactive Music Theory App}
%\def \CapstoneSponsorCompany{	Cheap Robots, Inc}
\def \CapstoneSponsorPerson{		Lukas Hein}

% 2. Uncomment the appropriate line below so that the document type works
\def \DocType{		Problem Statement
				%Requirements Document
				%Technology Review
				%Design Document
				%Progress Report
				}
			
\newcommand{\NameSigPair}[1]{\par
\makebox[2.75in][r]{#1} \hfil 	\makebox[3.25in]{\makebox[2.25in]{\hrulefill} \hfill		\makebox[.75in]{\hrulefill}}
\par\vspace{-12pt} \textit{\tiny\noindent
\makebox[2.75in]{} \hfil		\makebox[3.25in]{\makebox[2.25in][r]{Signature} \hfill	\makebox[.75in][r]{Date}}}}
% 3. If the document is not to be signed, uncomment the RENEWcommand below
%\renewcommand{\NameSigPair}[1]{#1}

%%%%%%%%%%%%%%%%%%%%%%%%%%%%%%%%%%%%%%%
\begin{document}
\begin{titlepage}
    \pagenumbering{gobble}
    \begin{singlespace}
    	\includegraphics[height=4cm]{coe_v_spot1}
        \hfill 
        % 4. If you have a logo, use this includegraphics command to put it on the coversheet.
        %\includegraphics[height=4cm]{CompanyLogo}   
        \par\vspace{.2in}
        \centering
        \scshape{
            \huge CS Capstone \DocType \par
            {\large\today}\par
            \vspace{.5in}
            \textbf{\Huge\CapstoneProjectName}\par
            \vfill
                %{\large Prepared for}\par
            %\Huge \CapstoneSponsorCompany\par
            \vspace{5pt}
            {\Large\NameSigPair{\CapstoneSponsorPerson}\par}
            {\large Prepared by }\par
            \GroupMemberOne\par
            % 5. comment out the line below this one if you do not wish to name your team
            %\CapstoneTeamName\par 
            \vspace{5pt}
            {\Large
                \NameSigPair{\GroupMemberOne}\par
            }
            \vspace{20pt}
        }
        \begin{abstract}
        % 6. Fill in your abstract
          Our project will be the creation of an iPhone application that helps beginner and intermediate students learn and develop new perspectives on music theory.
          The challenge is to present the core ideas of music theory in such a way that beginners can immediately use them to begin composing their own musical chord progressions. 
          This means presenting the concepts of intervals, 5\textsuperscript{th}s, notes and note names, the circle of 5\textsuperscript{th}s, chords, as well as some basic patterns for creating their own chord progressions.
          
          These can be presented by offering miniature lessons and explanations for each concept, interactive visualizations that allow students to ask and answer questions about how the theory relates to itself, the ability to compose music within the app and have the app verify that the music follows the composition rules, and challenges and games that help drill the students to put the concepts into their memory.

        \end{abstract}     
    \end{singlespace}
\end{titlepage}
\newpage
\pagenumbering{arabic}
\tableofcontents
% 7. uncomment this (if applicable). Consider adding a page break.
%\listoffigures
%\listoftables
\clearpage

% 8. now you write!
\section{Problem}
Music theory is a myriad of ideas, thoughts, notations, and lenses with which to analyze music. 
Currently, if you are a beginner in the world of music theory, the idea of composing or understanding the music you play and listen to is a Herculean task. 
Most discussions of music theory assumes that you know the fundamentals, but picking out which parts of music theory ARE the fundamentals is itself difficult. 
Even though there are progressions for teaching the fundamentals that are so simple that children as young as 8 years old can pick it up within a week, these progressions get lost in the confusion of all of the rest of music theory. 
On top of this, in order to be learned these methods need to be applied by students as they are taught, and students need to synthesize and compose their own music to stretch and solidify their understanding. 
Typically this means a teacher would be necessary to facilitate the student's learning.

\par
The method itself starts with a small core, and from there can be built on in layers. 
The core of the method is based on an interval of pitches called a 5\textsuperscript{th}. Because of the natural harmony between the pitch frequencies of the notes in a 5\textsuperscript{th}, the interval sounds very strongly connected in our ears. 
By following the intervals of 5\textsuperscript{th}s up or down, we walk through all of the notes in Western music, arriving at the note we started several octaves up or down. 
An octave is an interval that is so strongly connected in our ears that we perceive it as the same note. 
Since these 5\textsuperscript{th}s circle around all of the notes back to the starting note, we call this the circle of 5\textsuperscript{th}s. 

\par
Notes in a major scale can be determined from the circle of 5\textsuperscript{th}s by taking a specific subset of the adjacent notes of the tonic note you are looking at. 
These notes, and their ordering are used in the first method of composition. For example, the notes in C major scale are, BEADGCF (in this order). 
Compositions can be formed by following 3 rules:

\begin{enumerate}
\item Chords descend (go from left to right) by one step at a time
\item Chords can ascend (go from right to left) by one or more steps at a time
\item The final chord must be the tonic (C major, in this example).
\end{enumerate}

An example composition in C major might be: G Em Am Dm G C F G C
\clearpage

\section{Solution}
In order to present this method of teaching to beginners of music theory, we will be creating an interactive iPhone application.
At minimum, the app will present information about intervals, have an interactive visualization of the circle of 5\textsuperscript{th}s, and present the method of composition outlined above.
Interactive visuals of strings vibrating at different frequencies can be used to teach students about intervals.
Students can interact with the ratio of frequencies between strings and can choose different interesting musical intervals to learn more about how intervals work.
Note names will be displayed to help solidify new notations.

\par
The circle of 5\textsuperscript{th}s will be displayed in two ways simultaneously.
The traditional circle will be displayed, and can be spun around by students to see different notes at the top.
On the side, a vertical listing of the circle of fifths will be shown, with a window showing which notes are a part of the major scale.
This window can be drug up or down, which will spin the circle of fifths at the same time.
The student will have the option to show/hide relative minor keys as well.

\par
Students will be able to view the rules of composition outlined above.
In order to stretch their understanding they can enter in their own chord sequences, and have the app check whether or not their chord sequence matches the rules.

\par
There were other ideas for the app that were under discussion, but not yet nailed down.
\begin{itemize}
\item Allow users to play their compositions audibly
\item Allow for In-app purchases of course packs
\item Challenges to help students recognize when the rules of composition are being followed
\item Different methods of composition
\item Saving compositions
\item More chord types including inversions/voicings, 7ths, etc.
\item Combine keys to extend the number of chords in the original composition
\end {itemize}
\clearpage

\section{Performance Metrics}

The performance metrics are very minimal.
Lukas, the client, wants to take an iterative approach to developing the app.
He has lots of ideas for layers that he can add, but at the core he has the following requirements:
\begin{itemize}
\item Deployable to the iPhone or Android store
\item Displays the circle of 5\textsuperscript{th}s in the two ways outlined above
\item Displays the method of composition described above
\end {itemize}

\par
The performance metrics are very lenient, and I think that more discussions are needed with the client in order to lock these down to something more appropriate for the course.
Lukas is hesitant to ask for too much up front, since he is unaware of how much work it takes to make an app, and how to set his expectations.


\end{document}
